\documentclass[10pt]{beamer}
\synctex=1
\usepackage{amsmath,amsthm, amssymb, amsfonts, mathrsfs}
% \usepackage[pagebackref=true]{hyperref}
\usetheme[progressbar=frametitle,block=fill]{metropolis}
\metroset{block=fill}
\setbeamercolor{progress bar in head}{fg=red}
\usepackage{mathabx,epsfig}
\def\acts{\mathrel{\reflectbox{$\righttoleftarrow$}}}
\usepackage{cleveref, xcolor, bbm, array,
  esint,nicefrac,tikz-cd,tikz,enumitem}
% \usetikzlibrary{patterns}
\usepackage[toc,page]{appendix}
\usepackage{physics}
%\usepackage[protrusion=true,expansion=true]{microtype}
\usepackage{natbib}
% \usepackage[nottoc]{tocbibind}
% \usepackage{wrapfig}
\usepackage{float} 
% \usepackage{kpfonts}
% \usepackage[utf8]{inputenc}
% \usepackage[T1]{fontenc}
% \usepackage{yfonts}
% \usepackage{kpfonts,fontspec}
% \usepackage[math-style=ISO, bold-style=ISO]{unicode-math}

% \setmathfont{Garamond-Math.otf}

% \usepackage[urw-garamond]{mathdesign}
% \usepackage{garamondx}
% \usepackage{newtxmath}
%\usepackage{lmodern}

% \usepackage{CormorantGaramond}
%\setmathfont{Garamond-Math.otf}[StylisticSet={7,9}]

% \usepackage{minitoc, tocloft}

% \usepackage[margin=1in]{geometry}
%\usepackage[right= 4.5cm, left=4.5cm, top= 2.5cm, bottom=2.0cm]{geometry}

% \pagenumbering{gobble} 
% \setenumerate{noitemsep}


\let\mbb\mathbb
\let\mscr\mathscr
\let\mc\mathcal
\let\mf\mathfrak
\let\mbf\mathbf

\newcommand{\pur}[1]{{\color{purple!45!black} #1}}
\newcommand{\A}{\mathbb{A}}
\newcommand{\N}{\mathbb{N}}
\newcommand{\Z}{\mathbb{Z}}
\newcommand{\Q}{\mathbb{Q}}
\newcommand{\R}{\mathbb{R}}
\newcommand{\C}{\mathbb{C}}
\renewcommand{\P}{\mathbb{P}}
\renewcommand{\H}{\textfrak{H}}
\renewcommand{\O}{\mathscr{O}}
\newcommand{\Ca}{\mathscr{C}}
\newcommand{\F}{\mbb{F}}
\renewcommand{\L}{\mathscr{L}}
\newcommand{\p}{\mathfrak{p}}
\newcommand{\Lip}{\mathrm{Lip}\,}
\newcommand{\lav}{\mathrm{Lav}}
\newcommand{\Lav}{\mathrm{Lav}}
\newcommand{\loc}{\text{loc}}
\newcommand{\nsub}{\trianglelefteq}
\newcommand{\Set}{\mathbf{Set}}
\newcommand{\Ab}{\mathbf{Ab}}
\newcommand{\Sc}{\mathbf{Sc}}
\newcommand{\Aff}{\mathbf{Aff}}
\newcommand{\Coh}{\mathbf{Coh}}
\newcommand{\Ring}{\mathbf{Ring}}
\newcommand{\Comp}{\mathbf{Comp}}
\newcommand{\Mod}{\mathbf{Mod}}
\newcommand{\pow}[1]{[\![#1]\!]}
% \newcommand{\norm}[2]{\left\lVert #1 \right\rVert_{#2}}
\newcommand{\Norm}[1]{\left\lVert #1 \right\rVert}
\newcommand*{\defeq}{\mathrel{\vcenter{\baselineskip0.5ex \lineskiplimit0pt
      \hbox{\scriptsize.}\hbox{\scriptsize.}}}%
  =}

\newcommand\labelenumi{\textnormal{(\Roman{enumi})}}
\renewcommand\theenumi\labelenumi
\newcommand{\labelitemi}{$\circ$}
\newcommand{\labelitemii}{$\diamond$}
% \renewcommand{\qedsymbol}{$\blacksquare$ \\} 
\renewcommand{\setminus}{\smallsetminus}
\renewcommand{\bar}{\overline}
\renewcommand{\tilde}{\widetilde}
% \renewcommand{\phi}{\varphi}
\renewcommand{\theta}{\vartheta}
\renewcommand{\emptyset}{\varnothing}
\renewcommand{\op}{\text{op}}
\newcommand{\ls}[2]{\qty(\frac{#1}{#2})}

\renewcommand{\arraystretch}{2}
%\renewcommand{\cftsecfont}{\small}

\DeclareMathOperator{\supp}{supp}
\DeclareMathOperator{\Reg}{Reg}
% \DeclareMathOperator{\Res}
\DeclareMathOperator*{\esssup}{ess\,sup}
\DeclareMathOperator{\spn}{span}
\DeclareMathOperator{\Sl}{Sl}
\DeclareMathOperator{\Gl}{Gl}
\DeclareMathOperator{\Id}{Id}
\DeclareMathOperator{\Aut}{Aut}
\DeclareMathOperator{\Hom}{Hom}
\DeclareMathOperator{\Gal}{Gal}
\DeclareMathOperator{\Pic}{Pic}
\DeclareMathOperator{\Char}{char}
\DeclareMathOperator{\AC}{AC}
\DeclareMathOperator{\OP}{OP}
\DeclareMathOperator{\pSh}{pSh}
\DeclareMathOperator{\Sh}{Sh}
\DeclareMathOperator{\Spec}{Spec}
\DeclareMathOperator{\Fun}{Fun}
\DeclareMathOperator{\im}{Im}
\DeclareMathOperator{\et}{\acute{e}t}
\DeclareMathOperator{\Cl}{Cl}
% \theoremstyle{plain}
% \newtheorem{thm}{Theorem}[section]
% \newtheorem{cor}[thm]{Corollary}
% \newtheorem*{conj}{Question}
% \newtheorem{lemma}[thm]{Lemma}
% \newtheorem{prop}[thm]{Proposition}
% \theoremstyle{definition}
% \newtheorem{mydef}[thm]{Definition}
% \newtheorem{example}[thm]{Example}
% \newtheorem{non-example}[thm]{Non-example}
% \theoremstyle{remark}
% \newtheorem*{remark}{Remark}
% \theoremstyle{definition}
% \newtheorem{exercise}{Exercise}
% \numberwithin{equation}{section}

% \hypersetup{
%   colorlinks,
%   linkcolor={red!50!black},
%   citecolor={purple!45!black},
%   urlcolor={green!40!black}
% }
\makeatletter
\newcommand{\Pause}[1][]{\unless\ifmeasuring@\relax
\pause[#1]%
\fi}
\makeatother

\begin{document}
\title{Computing $p$-adic $L$-functions \\ \normalsize using Hilbert Eisenstein series}
\author{Håvard Damm-Johnsen}
\date{Y-RANT 18/08/21}
\institute{University of Oxford}

\maketitle
\begin{frame}
  \frametitle{Outline}

  \begin{itemize}[itemsep=5pt]
  \item \textbf{Motivation}: why study $L$-functions? 
    
  \item \textbf{Theory} behind the Klingen-Siegel method

  \item \textbf{Algorithm} for computing $p$-adic $L$-functions

  \item \textbf{Data}: some explicit $L$-functions 

  \item \textbf{Further research}: coming up
    
  \end{itemize}

  
\end{frame}



\begin{frame}
  \frametitle{Motivation: why do we care about $L$-functions?}
\begin{itemize}[leftmargin=2pt]\pause
\item \emph{$L$-functions}: $L(X,s)$ - meromorphic functions encoding
  arithmetic data\pause 
  \begin{itemize}
  \item $X = F$ number field: Dirichlet class number formula \pause
    \[ \Res_{s=1}\zeta_{F}(s)= \frac{2^{r_{1}}(2\pi)^{r_{2}}\Reg_{F}h_{F}
      }{w_{F}\sqrt{|D_{F}|}} \qq{where} \zeta_{F}(s) \defeq \sum_{I \subset \O_{F}}
      \frac{1}{N(I)^{s}}. \pause
    \]
  \item $X = $ elliptic curve: BSD conjecture, Gross-Zagier formula\pause
  \item $X = $ algebraic variety: Conjectures of Bloch, Beilinson, Deligne, ... \pause 
  \end{itemize}
\item \emph{$p$-adic $L$-functions}: 
  $L_{p}(X,s) \colon \Z_{p}\to \C_{p}$ - interpolates special values of
  $L(X,s)$ \pause 
\begin{itemize}
\item Iwasawa main conjectures \pause
\item $p$-adic BSD conjecture \pause
\end{itemize} 
\item \textbf{Data is important!}
\end{itemize}

\end{frame}
\begin{frame}
  \frametitle{Theory I}
\begin{itemize}\pause

\item \textbf{Euler + functional equation:} for $k$ even, $\zeta(1-k) = -B_{k}/k$.\pause
\item \textbf{Kummer:} if $k \equiv k' \mod{(p-1)p^{N}}$ are even, then
  \[   (1-p^{k-1})\frac{B_{k}}{k} \equiv (1-p^{k'-1})\frac{B_{k'}}{k'} \mod{p^{N}}.\pause
  \]
\vspace{-5pt}
\item By $p$-adic analysis, if
  $k_{0}\equiv k_{1}\equiv \ldots \mod{p-1}$, then $(1-p^{k_{i}-1})\zeta(1-k_{i})$ interpolate to
  $\zeta_{p}\colon \Z_{p}\to \C_{p}$. \pause
\item different ``branches'' corresponding to $\bar k_{0} \in \Z/(p-1)\Z$.\pause
\item We can approximate $\zeta_{p}(s)$ by computing $\zeta(1-k)$ for
  enough $k$ and use polynomial interpolation: \pause

\item given pairs
  $(1-k_{0},\zeta_{p}(1-k_{0})),\ldots, (1-k_{d},\zeta_{p}(1-k_{d}))$, find a polynomial
  $P$ of degree $d$ such that $P(1-k_{i}) = \zeta_{p}(1-k_{i})$. \pause
\item \textbf{Eg.} $p=5, k_{i}\in \{2,6,10\}$, interpolate $\{(k_{i},-(1-5^{k_{i}-1})B_{k_{i}}/k_{i})\}$: \pause
  \[\left(5^{2} \cdot 4828 + O(5^{8})\right) s^{2} + \left(5 \cdot 60514 + O(5^{8})\right) s + 51662 + O(5^{8})\pause \ ``\approx"  \ \zeta_{5}(s)
  \]
\end{itemize}

\end{frame}




\begin{frame}
  \frametitle{Theory II}
\begin{itemize}[leftmargin=2pt]\pause
\item Fix $F$ a totally real number field of degree $d$, $\psi$ a
  character of $\Cl_{\mf m}^{+}$. \pause
\item Define $L(\psi,s) \defeq \sum_{n > 0} \psi(n)n^{-s}$; \pause then
  $L(\psi,1-k) = -B_{k,\psi}/k \in \bar \Q$.\pause
\item Interpolation: $L_{p}(\psi,1-k) = (1-\psi(p)p^{k-1})L(\psi,1-k)$. \pause
\item However, not obvious how to compute $B_{k,\psi}$ algebraically, so
  we need a new strategy! \pause
\item \textbf{Serre (Antwerp III):} The modular form
  \[ G_{k}^{*} = \frac{1}{2} \zeta_{p}(1-k) + \sum_{n=1}^{\infty}\qty(\sum_{d \mid n,\,
    (d,p)=1}d^{k-1})q^{n}
  \]
  on $\Gamma_{0}(p)$ is a $p$-adic limit of $G_{k_{i}}$. \pause
\item Congruences between non-constant terms gives congruences between
  the constant terms, in fact, the Kummer congruences.
\end{itemize}
% (More details can be found in a preprint by Lauder-Vonk)
\end{frame}

\begin{frame}{Theory III}
\begin{itemize}[leftmargin=-3pt]
\item \textbf{Idea:} look for modular forms with constant term $L(\psi,1-k)$.\pause
% \item  A \textbf{Hilbert modular form} is a function $f \colon \mf h^{d} \to
%   \C$ which ``transforms like a modular form'' under
%   $\Sl_{2}(\O_{F}) \acts \mf h^{d}$.\pause
\item There is a \emph{Hilbert modular form}
  \[ \hspace{-10pt}G_{k,\psi}^{(p)}(z_{1},\ldots, z_{d}) = L_{p}(\psi,1-k) + 2^{d}\sum_{\nu \in \mf
      d^{-1}_{+}}\Big(\!\sum_{\substack{\mf a \mid (\nu)\mf d \\ (\mf a, p) =
        1}}\psi(\mf a)N(\mf a)^{k-1}\!\Big)e^{2\pi i (\sigma_{1}(\nu)z_{1} + \ldots
      \sigma_{d}(\nu)z_{d})}. \pause
  \]
  
% \item \textbf{Deligne-Ribet:} the ``Hilbert Eisenstein series'',
%   \[ \hspace{-10pt}G_{k,\psi}^{(p)}(z_{1},\ldots, z_{d}) = L_{p}(\psi,1-k) + 2^{d}\sum_{\nu \in \mf
%       d^{-1}}\Big(\!\sum_{\substack{\mf a \mid (\nu)\mf d \\ (\mf a, p) =
%         1}}\psi(\mf a)N(\mf a)^{k-1}\!\Big)e^{2\pi i (\sigma_{1}(\nu)z_{1} + \ldots \sigma_{d}(\nu)z_{d})}
%   \]

\item Setting $z_{1} = \ldots = z_{d} = z$, we get the \textbf{diagonal restriction}
  \[\Delta_{k,\psi}(z) = L_{p}(\psi,1-k) + 2^{d}\sum_{n>0}\sum_{\substack{ \nu \in \mf
      d^{-1}_{+}\\ \tr(\nu) = n}}\Big(\!\sum_{\substack{\mf a \mid (\nu)\mf d \\ (\mf a, p) =
        1}}\psi(\mf a)N(\mf a)^{k-1}\!\Big)q^{n}\pause
  \]
\item This is a classical modular form of weight $dk$. \pause
\item \textbf{Key idea:} knowing a basis for $M_{dk}$ and enough coefficients of
  $\Delta_{k,\psi}(z)$ lets us solve for $L_{p}(\psi,1-k)$!
\end{itemize}

\end{frame}

\begin{frame}
  \frametitle{Algorithm}\pause
  \textbf{Input:}\pause
  \begin{itemize}[itemsep=0pt,leftmargin=2pt]
  \item $F$ totally real number field,\pause
  \item $p$ odd prime,\pause
  \item $k_0 \in \Z $ ``starting weight'', $m \in \N$ ``precision'',\pause
  \item  $\psi$ a character of $F$ with the same parity as $k_{0}$. \pause
  \end{itemize}

 \textbf{Output:} $L_p(\psi,s)$ as an element of $\O_{F}[s]/(p^m)$. \pause

 \begin{enumerate}[leftmargin=2pt]
 \item Set interpolation weights $k_{j} = k_{0} + j(p-1)$, \pause  compute
   bases for $M_{dk_{j}}(\Gamma_{0}(M),\psi)$ for $j =1, \ldots ,\delta_{m}$.\pause
 \item Compute non-constant coefficients of $\Delta_{k_{j},\psi}$:
   \[ a_{n} =\sum_{\substack{ \nu \in \mf
      d^{-1}_{+}\\ \tr(\nu) = n}}\sum_{\substack{\mf a \mid (\nu)\mf d \\ (\mf a, p) =
        1}}\psi(\mf a)N(\mf a)^{k_{j}-1}, \qq{} n>0. \pause
   \]
 \item Write $\Delta_{k_{j},\psi}$ as linear combination of modular forms in
   $M_{dk_{j}}$ and solve for constant term, $L_{p}(\psi,1-k_{j})$. \pause
 \item Interpolate the $\delta_{m}+1$ values of $L_{p}(\psi,1-k_{j})$ to find
   $L_{p}(\psi,s)$.
 \end{enumerate}

\end{frame}

% \begin{frame}
%   \frametitle{Data, $F = \Q(\sqrt{3})$}
%   \tiny
% \begin{tabular}{|p{.02\linewidth}|p{.93\linewidth}|} \hline
% {\small $p$ }&{\small  $L_p(\mbf 1,s) \in \O_{F}[s] \pmod{p^{m}}$} \\ \hline
%   \hline
% $5$ & $O(5^{8}) s^{14} + O(5^{8}) s^{13} + O(5^{8}) s^{12} + O(5^{8}) s^{11} + O(5^{8}) s^{10} + O(5^{8}) s^{9} + O(5^{8}) s^{8} + \left(5^{6} \cdot 12 + O(5^{8})\right) s^{7} + \left(5^{6} \cdot 18 + O(5^{8})\right) s^{6} + \left(5^{4} \cdot 308 + O(5^{8})\right) s^{5} + \left(5^{5} \cdot 12 + O(5^{8})\right) s^{4} + \left(5^{3} \cdot 811 + O(5^{8})\right) s^{3} + \left(5^{3} \cdot 1839 + O(5^{8})\right) s^{2} + \left(5 \cdot 202997 + O(5^{9})\right) s + 8760231 + O(5^{10})$ \\ \hline
% $7$ & $O(7^{9}) s^{12} + O(7^{9}) s^{11} + O(7^{9}) s^{10} + \left(7^{8} \cdot 4 + O(7^{9})\right) s^{9} + \left(7^{7} \cdot 26 + O(7^{9})\right) s^{8} + \left(7^{6} \cdot 339 + O(7^{9})\right) s^{7} + \left(7^{6} \cdot 41 + O(7^{9})\right) s^{6} + \left(7^{5} \cdot 1955 + O(7^{9})\right) s^{5} + \left(7^{4} \cdot 12339 + O(7^{9})\right) s^{4} + \left(7^{3} \cdot 43352 + O(7^{9})\right) s^{3} + \left(7^{2} \cdot 755539 + O(7^{9})\right) s^{2} + \left(7 \cdot 2363400 + O(7^{9})\right) s + 236098386 + O(7^{10})$ \\ \hline
% $11$ & $O(11^{9}) s^{12} + O(11^{9}) s^{11} + O(11^{10}) s^{10} + \left(11^{9} \cdot 7 + O(11^{10})\right) s^{9} + \left(11^{9} \cdot 9 + O(11^{10})\right) s^{8} + \left(11^{7} \cdot 983 + O(11^{10})\right) s^{7} + \left(11^{6} \cdot 2077 + O(11^{10})\right) s^{6} + \left(11^{5} \cdot 99624 + O(11^{10})\right) s^{5} + \left(11^{4} \cdot 874530 + O(11^{10})\right) s^{4} + \left(11^{3} \cdot 15178197 + O(11^{10})\right) s^{3} + \left(11^{2} \cdot 760252 + O(11^{9})\right) s^{2} + \left(11 \cdot 125621362 + O(11^{9})\right) s + 18201420921 + O(11^{10})$ \\ \hline
% $13$ & $O(13^{10}) s^{11} + O(13^{10}) s^{10} + \left(13^{9} \cdot 10 + O(13^{10})\right) s^{9} + \left(13^{8} \cdot 118 + O(13^{10})\right) s^{8} + \left(13^{7} \cdot 811 + O(13^{10})\right) s^{7} + \left(13^{6} \cdot 210 + O(13^{10})\right) s^{6} + \left(13^{5} \cdot 88733 + O(13^{10})\right) s^{5} + \left(13^{4} \cdot 3403408 + O(13^{10})\right) s^{4} + \left(13^{4} \cdot 2585989 + O(13^{10})\right) s^{3} + \left(13^{2} \cdot 84148803 + O(13^{10})\right) s^{2} + \left(13 \cdot 9530516807 + O(13^{10})\right) s + 41080902718 + O(13^{10})$ \\ \hline
% $17$ & $O(17^{10}) s^{11} + O(17^{10}) s^{10} + \left(17^{9} \cdot 10 + O(17^{10})\right) s^{9} + \left(17^{8} \cdot 277 + O(17^{10})\right) s^{8} + \left(17^{7} \cdot 2442 + O(17^{10})\right) s^{7} + \left(17^{6} \cdot 20929 + O(17^{10})\right) s^{6} + \left(17^{5} \cdot 144868 + O(17^{10})\right) s^{5} + \left(17^{4} \cdot 16905741 + O(17^{10})\right) s^{4} + \left(17^{3} \cdot 15764965 + O(17^{10})\right) s^{3} + \left(17^{2} \cdot 4172504534 + O(17^{10})\right) s^{2} + \left(17^{2} \cdot 6858913505 + O(17^{10})\right) s + 1413992627558 + O(17^{10})$ \\ \hline
% $19$ & $O(19^{10}) s^{11} + O(19^{10}) s^{10} + \left(19^{9} \cdot 5 + O(19^{10})\right) s^{9} + \left(19^{8} \cdot 96 + O(19^{10})\right) s^{8} + \left(19^{7} \cdot 2895 + O(19^{10})\right) s^{7} + \left(19^{6} \cdot 56680 + O(19^{10})\right) s^{6} + \left(19^{5} \cdot 2164758 + O(19^{10})\right) s^{5} + \left(19^{4} \cdot 9456125 + O(19^{10})\right) s^{4} + \left(19^{3} \cdot 809885956 + O(19^{10})\right) s^{3} + \left(19^{2} \cdot 10906738871 + O(19^{10})\right) s^{2} + \left(19 \cdot 19464966193 + O(19^{10})\right) s + 6019547436581 + O(19^{10})$ \\ \hline
% \end{tabular}
% \end{frame}

\begin{frame}
  \frametitle{Data, $F = \Q(\sqrt{7})$}
  \tiny
\begin{tabular}{|p{.02\linewidth}|p{.95\linewidth}|} \hline
{\small $p$} & {\small $L_p(\mbf{1},s) \in \O_{F}[s] \pmod{p^{m}}$} \\ \hline \hline
$5$ & $O(5^{8}) s^{14} + O(5^{8}) s^{13} + O(5^{8}) s^{12} + O(5^{8}) s^{11} + O(5^{8}) s^{10} + O(5^{8}) s^{9} + \left(5^{7} \cdot 1 + O(5^{8})\right) s^{8} + \left(5^{6} \cdot 2 + O(5^{8})\right) s^{7} + \left(5^{5} \cdot 49 + O(5^{8})\right) s^{6} + \left(5^{4} \cdot 568 + O(5^{8})\right) s^{5} + \left(5^{4} \cdot 186 + O(5^{8})\right) s^{4} + \left(5^{3} \cdot 2476 + O(5^{8})\right) s^{3} + \left(5^{2} \cdot 12643 + O(5^{8})\right) s^{2} + \left(5 \cdot 116522 + O(5^{9})\right) s + 1005394 + O(5^{10})$ \\ \hline
$11$ & $O(11^{9}) s^{12} + O(11^{9}) s^{11} + O(11^{10}) s^{10} + \left(11^{9} \cdot 9 + O(11^{10})\right) s^{9} + \left(11^{8} \cdot 45 + O(11^{10})\right) s^{8} + \left(11^{7} \cdot 538 + O(11^{10})\right) s^{7} + \left(11^{6} \cdot 5908 + O(11^{10})\right) s^{6} + \left(11^{5} \cdot 94233 + O(11^{10})\right) s^{5} + \left(11^{4} \cdot 653451 + O(11^{10})\right) s^{4} + \left(11^{3} \cdot 1368033 + O(11^{10})\right) s^{3} + \left(11^{3} \cdot 691404 + O(11^{9})\right) s^{2} + \left(11 \cdot 43622653 + O(11^{9})\right) s + 25656523351 + O(11^{10})$ \\ \hline
$13$ & $O(13^{10}) s^{11} + O(13^{10}) s^{10} + \left(13^{9} \cdot 9 + O(13^{10})\right) s^{9} + \left(13^{8} \cdot 167 + O(13^{10})\right) s^{8} + \left(13^{7} \cdot 825 + O(13^{10})\right) s^{7} + \left(13^{6} \cdot 20775 + O(13^{10})\right) s^{6} + \left(13^{5} \cdot 260717 + O(13^{10})\right) s^{5} + \left(13^{4} \cdot 3958931 + O(13^{10})\right) s^{4} + \left(13^{3} \cdot 10298345 + O(13^{10})\right) s^{3} + \left(13^{3} \cdot 37593275 + O(13^{10})\right) s^{2} + \left(13 \cdot 10196962616 + O(13^{10})\right) s + 104887446825 + O(13^{10})$ \\ \hline
$17$ & $O(17^{10}) s^{11} + O(17^{10}) s^{10} + \left(17^{9} \cdot 16 + O(17^{10})\right) s^{9} + \left(17^{9} \cdot 11 + O(17^{10})\right) s^{8} + \left(17^{7} \cdot 3442 + O(17^{10})\right) s^{7} + \left(17^{6} \cdot 43576 + O(17^{10})\right) s^{6} + \left(17^{5} \cdot 731121 + O(17^{10})\right) s^{5} + \left(17^{4} \cdot 19535454 + O(17^{10})\right) s^{4} + \left(17^{3} \cdot 2220157 + O(17^{10})\right) s^{3} + \left(17^{2} \cdot 311956925 + O(17^{10})\right) s^{2} + \left(17 \cdot 12743287888 + O(17^{10})\right) s + 497360978290 + O(17^{10})$ \\ \hline
$19$ & $O(19^{10}) s^{11} + O(19^{10}) s^{10} + \left(19^{9} \cdot 3 + O(19^{10})\right) s^{9} + \left(19^{8} \cdot 356 + O(19^{10})\right) s^{8} + \left(19^{7} \cdot 5512 + O(19^{10})\right) s^{7} + \left(19^{6} \cdot 86567 + O(19^{10})\right) s^{6} + \left(19^{5} \cdot 784303 + O(19^{10})\right) s^{5} + \left(19^{4} \cdot 35196026 + O(19^{10})\right) s^{4} + \left(19^{3} \cdot 755707686 + O(19^{10})\right) s^{3} + \left(19^{2} \cdot 13133906787 + O(19^{10})\right) s^{2} + \left(19 \cdot 27470894456 + O(19^{10})\right) s + 226617386081 + O(19^{10})$ \\ \hline
\end{tabular}

\end{frame}
\begin{frame}
  \frametitle{Implementation}
\begin{itemize}[leftmargin=2pt]\pause
\item Currently only implemented for trivial character, real quadratic fields.\pause
\item \texttt{sage} has experimental support for Hecke characters via 
  \texttt{gp/pari}.\pause
% \item \texttt{magma} implementation only works with ring class
%   characters.\pause
\item Computationally heavy bits:\pause
  \begin{itemize}
  \item Computing diagonal restriction coefficients\pause
    
  \item[->] For ring class characters, reduction theory of quadratic forms\pause
  \item Finding $q$-expansion bases of modular forms of high weight\pause
  \item[->] Randomised algorithm due to Lauder \pause
  \item[->] Is highly parallelisable \pause
  \end{itemize}
\item Roblot '15: Alternative algorithm based on Cassou-Noguès'
  construction of $p$-adic $L$-functions - less efficient in practice.
\end{itemize}

\end{frame}



\begin{frame}
  \frametitle{Further venues}
  \begin{itemize}[leftmargin=2pt]
  \item Statistics of $\lambda$-invariants à la Ellenberg-Jain-Venkatesh.\pause
\item Darmon-Pozzi-Vonk: if $L(\psi,s)$ has an exceptional zero at $0$, then $L'(\psi,0)$ is the
  constant term of an overconvergent modular form\pause
  \begin{itemize}
  \item computed using a similar algorithm\pause
  \item gives log of ``Gross-Stark units'' of the Hilbert class
    field of $F$\pause
  \end{itemize}
\item Can we generalise to $p$-adic
  $L$-functions of automorphic representations?
\end{itemize}

\end{frame}
\begin{frame}[standout]
  Thank you! \\ 
  Questions?
\end{frame}


\end{document}

%%% Local Variables:
%%% mode: latex
%%% TeX-master: t
%%% End:
