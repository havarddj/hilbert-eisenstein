\documentclass[11pt,a4paper]{article}

\synctex=1
\usepackage{amsmath,amsthm, amssymb, amsfonts, mathrsfs,
  enumitem,adjustbox,caption,setspace,xspace}
\usepackage[pagebackref=true]{hyperref}

\usepackage[ruled,vlined]{algorithm2e}
\SetKwInput{KwInput}{Input}
\SetKwInput{KwOutput}{Output}

\usepackage{xcolor, bbm, rotating, array,
  esint,nicefrac}
\usepackage{cleveref}
% Tikz stuff
\usepackage{tikz-cd,tikz}
% \usepackage{pgfplots}

% \usetikzlibrary{patterns}
% \usetikzlibrary{decorations.pathmorphing}
% \usetikzlibrary{decorations.markings}
% \usetikzlibrary{arrows.meta,bending}
% \tikzset{cong/.style={draw=none,edge node={node [sloped, allow upside down, auto=false]{$\cong$}}},
% Isom/.style={draw=none,every to/.append style={edge node={node [sloped, allow upside down, auto=false]{$\cong$}}}}}

% % \usepackage{pst-plot}
% % \usepackage{auto-pst-pdf}

% 
\usepackage[toc,page]{appendix}

% \usepackage[protrusion=true,expansion=true]{microtype}
\usepackage{natbib}
% \citestyle{apalike}
\usepackage[nottoc]{tocbibind}
% \usepackage{wrapfig}
\usepackage{float}
\usepackage[stable]{footmisc}
% \usepackage{kpfonts}
% \usepackage[utf8]{inputenc}
% \usepackage[T1]{fontenc}
\usepackage{yfonts}
\usepackage{kpfonts,fontspec}
\usepackage[math-style=ISO, bold-style=ISO]{unicode-math}

\setmathfont{Garamond-Math.otf}
% \usepackage[urw-garamond]{mathdesign}
\usepackage{garamondx}
\tikzcdset{arrow style=tikz}
% \usepackage{newtxmath}
% \usepackage{lmodern}

% \usepackage{CormorantGaramond}
% \setmathfont{Garamond-Math.otf}[StylisticSet={7,9}]

\usepackage{tocloft}
\usepackage{etoc}
\usepackage{physics}
% \usepackage[margin=1in]{geometry}
% \usepackage[right= 4.5cm, left=4.5cm, top= 2.5cm, bottom=2.5cm]{geometry}

% \pagenumbering{gobble} 
\setenumerate{noitemsep}

\let\mbb\mathbb
\let\mscr\mathscr
\let\mc\mathcal
\let\mbf\mathbf
\let\mf\mathfrak
\newcommand{\A}{\mathbb{A}}
\newcommand{\N}{\mathbb{N}}
\newcommand{\Z}{\mathbb{Z}}
\newcommand{\Q}{\mathbb{Q}}
\newcommand{\R}{\mathbb{R}}
\newcommand{\G}{\mbb G}
\newcommand{\C}{\mathbb{C}}
\renewcommand{\P}{\mathbb{P}}
% \newcommand{\im}{\mathrm{Im}}
% \newcommand{\re}{\mathrm{Re}}
\renewcommand{\H}{\textfrak{H}}
\renewcommand{\O}{\mathscr{O}}
\newcommand{\Ca}{\mathscr{C}}
\newcommand{\F}{\mbb{F}}
\renewcommand{\L}{\mathscr{L}}
\newcommand{\p}{\mathfrak{p}}
\newcommand{\Lip}{\mathrm{Lip}\,}
\newcommand{\Jap}[1]{\langle #1 \rangle}
\newcommand{\ps}[1]{[\![#1]\! ]}
\newcommand{\lav}{\mathrm{Lav}}
\newcommand{\Lav}{\mathrm{Lav}}
\newcommand{\loc}{\text{loc}}
\newcommand{\nsub}{\trianglelefteq}
\newcommand{\Set}{\mathbf{Set}}
\newcommand{\Ab}{\mathbf{Ab}}
\newcommand{\Sc}{\mathbf{Sc}}
\newcommand{\MHS}{\mathbf{MHS}}
\newcommand{\Aff}{\mathbf{Aff}}
\newcommand{\Coh}{\mathbf{Coh}}
\newcommand{\Ring}{\mathbf{Ring}}
\newcommand{\Comp}{\mathbf{Comp}}
\newcommand{\dR}{\mathrm{dR}}
% \newcommand{\Mod}{\mathbf{Mod}}
\newcommand{\shHom}{\mscr{H}\!\!\mscr{om}}
\newcommand{\shExt}{\mscr{Ext}}
\renewcommand{\textnumero}{N\textsuperscript{o}\xspace}

\newcommand{\pow}[1]{[\![#1]\!]}
% \newcommand{\norm}[2]{\left\lVert #1 \right\rVert_{#2}}
\newcommand{\Norm}[1]{\left\lVert #1 \right\rVert}
\newcommand*{\defeq}{\mathrel{\vcenter{\baselineskip0.5ex \lineskiplimit0pt
      \hbox{\scriptsize.}\hbox{\scriptsize.}}}%
  =}

\renewcommand\labelenumi{\textnormal{(\roman{enumi})}}
\renewcommand\theenumi\labelenumi

% \renewcommand{\qedsymbol}{$\blacksquare$ \\} 
\renewcommand{\setminus}{\smallsetminus}
\renewcommand{\bar}{\overline}
\renewcommand{\tilde}{\widetilde}
\renewcommand{\phi}{\varphi}
\renewcommand{\theta}{\vartheta}
\renewcommand{\emptyset}{\varnothing}
\renewcommand{\op}{\text{op}}
\newcommand{\ls}[2]{\qty(\frac{#1}{#2})}
% \renewcommand\labelitemi{$\circ$}
% \renewcommand{\arraystretch}{2}
% \renewcommand{\cftsecfont}{\small}

\DeclareMathOperator{\supp}{supp}
% \DeclareMathOperator{\ord}{ord}
\DeclareMathOperator{\cd}{cd}
\DeclareMathOperator{\an}{an}
\DeclareMathOperator{\sep}{sep}
\DeclareMathOperator{\Nm}{Nm}
\DeclareMathOperator{\res}{res}
\DeclareMathOperator{\alg}{alg}
\DeclareMathOperator{\Alg}{Alg}
\DeclareMathOperator{\gr}{gr}
\DeclareMathOperator{\Zar}{Zar}
\DeclareMathOperator*{\esssup}{ess\,sup}
\DeclareMathOperator{\spn}{span}
\DeclareMathOperator{\Sl}{Sl}
\DeclareMathOperator{\Gl}{Gl}
\DeclareMathOperator{\Cl}{Cl}
\DeclareMathOperator{\pr}{pr}
\DeclareMathOperator{\Id}{Id}
\DeclareMathOperator{\Aut}{Aut}
\DeclareMathOperator{\Hom}{Hom}
\DeclareMathOperator{\Ext}{Ext}
\DeclareMathOperator{\Mod}{Mod}
\DeclareMathOperator{\Gal}{Gal}
\DeclareMathOperator{\Et}{Ét}
\DeclareMathOperator{\et}{ét}
\DeclareMathOperator{\Br}{Br}
\DeclareMathOperator{\Restr}{res}
\DeclareMathOperator{\Op}{Op}
\DeclareMathOperator{\Char}{char}
\DeclareMathOperator{\Div}{Div}
\DeclareMathOperator{\AC}{AC}
\DeclareMathOperator{\OP}{OP}
\DeclareMathOperator{\pSh}{pSh}
\DeclareMathOperator{\Sh}{Sh}
\DeclareMathOperator{\Sch}{Sch}
\DeclareMathOperator{\Spec}{Spec}
\DeclareMathOperator{\Frac}{Frac}
\DeclareMathOperator{\Pic}{Pic}
\DeclareMathOperator{\Jac}{Jac}
\DeclareMathOperator{\Fun}{Fun}
\DeclareMathOperator{\im}{Im}
\DeclareMathOperator{\re}{Re}
\DeclareMathOperator{\coker}{coker}

\theoremstyle{plain}
\newtheorem{thm}{Theorem}[section]
\newtheorem*{thm*}{Theorem}
\newtheorem{cor}[thm]{Corollary}
\newtheorem*{conj}{Question}
\newtheorem{lemma}[thm]{Lemma}
\newtheorem{prop}[thm]{Proposition}
\theoremstyle{definition}
\newtheorem{mydef}[thm]{Definition}
\newtheorem{eg}[thm]{Example}
\newtheorem{non-example}[thm]{Non-example}
\theoremstyle{remark}
\newtheorem*{remark}{Remark}
\newtheorem*{exercise}{Exercise}

\numberwithin{equation}{section}
\setlength{\parskip}{.5em}
% \setlength{\parindent}{16.5pt}

\colorlet{phoen}{red!50!black}
\colorlet{purpur}{purple!45!black}
\colorlet{skog}{green!30!black}
\colorlet{hav}{blue!30!black}
\colorlet{himmel}{purpur!90!white}
\hypersetup{
  colorlinks,
  linkcolor={red!50!black},
  citecolor={purple!45!black},
  urlcolor={green!40!black}
}


\begin{document}
\title{\vspace{-1cm} $p$-adic $L$-functions à la Klingen-Siegel}
\author{H\aa vard Damm-Johnsen}
\date{September 2021}
\maketitle
% \renewcommand{\abstractname}{Introduction}
% \begin{abstract}

% \end{abstract}
\localtableofcontents

\section{Introduction}
\label{sec:introduction}
$L$-functions are ubiquitous in modern number theory, and their
behaviour is described by a number of important conjectures which have
been verified in only a very limited number of cases. While computer
calculation might be considered a good tool for testing the
conjectures, historically some of their formulations -- most notably the
Birch--Swinnerton-Dyer conjecture -- were directly inspired by numerical
observations. 
\subsection{Historical background}
\label{sec:background}
The history of $p$-adic $L$-function can arguably be traced to
Kummer's work on Fermat's last theorem. By studying class groups of
cyclotomic fields, he found the surprising congruence
\begin{equation}
  \label{eq:8}
  (1-p^{k-1})\frac{B_{k}}{k} \equiv   (1-p^{k'-1})\frac{B_{k'}}{k'} \mod{p^{m+1}}
\end{equation}
when $k$ and $k'$ are even integers satisfying $k \equiv k' \mod{(p-1)p^{m}}$.
On the other hand, by Euler's solution to the Basel problem and the
functional equation for the Riemann zeta function, we have
\begin{equation}
  \label{eq:9}
  \zeta(1-k) =- \frac{B_{k}}{k}, \qq{} k \text{ even,}
\end{equation}
leading Kubota and Leopoldt to interpret \cref{eq:8} as a congruence
between special values of $\zeta$. A density argument then gives a
$p$-adic analytic function $\zeta_{p}\colon \Z_{p}\to \C_{p}$ which
``interpolates'' $\zeta$ in the sense that
\begin{equation}
  \label{eq:10}
  \zeta_{p}(1-k) = (1-p^{1-k})\zeta(1-k), \qq{} k\text{ even}.
\end{equation}
The appearance of the factor $(1-p^{1-k})$ is conceptually explained
as dividing out by the $p$-th Euler factor in the Euler product of
$\zeta$.

To be precise, we actually obtain a \emph{collection} of analytic
functions $\zeta_{p,i}(s)$ where $i\in (\Z/p\Z)^{\times}$ is the congruence class
of the integers $k$ from which $\zeta_{p,i}$ is interpolated. This is
often referred to as the $i$-th Iwasawa branch of $\zeta_{p}$. 

It is worth briefly mentioning a more conceptual construction of $\zeta_{p}$:
recall from the proof of the functional equation of $\zeta$ that
$\zeta$ is, up to a factor of $(s-1)$, the \emph{Mellin transform} of
$f(t) = \frac{t}{e^{t}-1}$. This is the so-called ``exponential generating function'' of the
Bernoulli numbers, meaning we have a formal expansion
\begin{equation}
  \label{eq:12}
  \frac{t}{e^{t}-1} = \sum_{k=1}^{\infty}\frac{B_{k}t^{k}}{k!}.
\end{equation}

Analogously we can define $\zeta_{p}$ as a
``$p$-adic Mellin transform'' of a suitably chosen
``pseudo-measure''. This ties into the interesting philosophy of
treating analysis over the various completions of $\Q$ on equal
footing, in contrast with the historical preference for the
archimedean completions.

Dirichlet defined, as part of his work on primes in arithmetic
progressions, $L$-functions attached to Dirichlet
characters\footnote{Of course, the name came later.}. More precisely,
let $\chi \colon (\Z/N\Z)^{\times}\to \C^{\times}$ be a group homomorphism and extend to
$\Z$ by the rule $\chi(p) = 0$ for $p \mid N$. Then we can define the
\emph{Dirichlet $L$-function}
\begin{equation}
  \label{eq:11}
 L(\chi,s) \defeq  \sum_{n =1}^{\infty}\frac{\chi(n)}{n^{-s}}, \qq{} \Re(s) > 1,
 \end{equation}
 and in a manner similar to $\zeta$ show that $L$ admits an Euler product
 and a functional equation. In particular, $L(\chi,s)$ is the Mellin
 transform of $\sum_{a=1}^{N}\frac{te^{at}}{e^{Nt}-1}$, which is the
 generating function of the so-called \emph{generalised Bernoulli
   numbers}, $B_{k,\chi}$.

 It is natural to ask whether these give rise
 to a $p$-adic analytic function like $\zeta_{p}$, and indeed we can
 construct the $p$-adic $L$-function $L_{p}(\chi,s)$ simply by twisting
 the measure defining $\zeta_{p}$ by $\chi$. In particular, $\zeta_{p,i}(s) =
 L_{p}(\omega^{i},s)$ where $\omega\colon \Z/(p-1)\Z \to \Z_{p}$ is the
 Teichmüller character. This also features in the special value
 formula
 \begin{equation}
   \label{eq:13}
   L_{p}(\chi,1-k) = (1-\chi\omega^{-k}(p)p^{k-1})L(\chi\omega^{-k},1-k), \qq{} k \geq 1.
 \end{equation}

 \subsection{Overview}
 \label{sec:overview}
In XYZ we do ABC etc.

\section{Theory}
\label{sec:Theory}

\subsection{Hecke characters}
The main reference for this section is \cite{miyake1989}.

When extending the definition of the Riemann zeta function $\zeta$ to
general number fields $F$, i.e. the Dedekind zeta function
$\zeta_{F}$, one immediately finds that the correct generalisation is to
replace the summation over the integers in $\zeta$ to summation over
integral ideals in $\O_{F}$, and then taking norms. If we would like
to define an analogue of the Dirichlet $L$-functions to $F$, then it
is necessary to define characters whose domain is the set of
(integral) ideals of $F$, which is precisely what lead Hecke to define
his Grössencharaktere:

  Let $\mf m$ be an integral ideal of $F$ and set
  \begin{equation}
    \label{eq:17}
     J_{\mf m} \defeq \{ \mf a \leq \O_{F} : (\mf a, \mf m) = 1\}
     \qq{and} P_{\mf m} \defeq \{ (a) \leq \O_{F} : F^{\times} \ni a \equiv 1 \mod^{\times}\mf m\}.
   \end{equation}
   Here $a \equiv 1 \mod^{\times}$ means that we can find
   $b,c \in \O_{F}^{\times}$ coprime to $\mf m$ such that $a = b/c$ and $b \equiv
   c \mod \mf m$. Enumerate the $r_{1}$ real embeddings of $F$ into $\C$ up
   to conjugation as $\sigma_{\nu}$ for $1 \le \nu \le r_{1}$ and the $r_{2}$ imaginary
   embeddings modulo conjugation
   as $\sigma_{\nu}$ for $r_{1} \le \nu \le r_{1}+r_{2}$. 
\begin{mydef}
  A \textbf{Hecke character modulo $\mf m$} is a group homomorphism $\chi \colon J_{\mf
    m}\to \C^{\times}$ satisfying for any $(a) \in P_{\mf m}$,
  \begin{equation}
    \label{eq:19}
    \chi((a)) = \prod_{\nu =1}^{r_{1}+r_{2}}\qty(\frac{\sigma_{\nu}(a)}{|\sigma_{\nu}(a)|})^{u_{\nu}} |\sigma_{\nu}|^{iv_{\nu}},
  \end{equation}
  where $u_{\nu} \in \{0,1\}$ for $1 \le \nu\le r_{1}$ and $u_{\nu} \in \Z$
  otherwise, and $v_{\nu}\in \R$ satisfy
  \begin{equation}
    \label{eq:20}
    \sum_{\nu=1}^{r_{1}+r_{2}} u_{\nu} = 0.
  \end{equation}
  The tuple $(u_{\nu})$ is called the \textbf{infinity-type} of $\chi$.
\end{mydef}
Suppose $\chi$ is a Hecke character of modulus $\mf m$. If
$\mf m \mid \mf m'$, then $\chi$ is naturally a Hecke character of modulus
$\mf m'$ by restricting its domain to $J_{\mf m'}$. 
\begin{mydef}
  The minimal $\mf m$ for which $\chi$ is a Hecke character of modulus
  $\mf m$ is called the
  \textbf{conductor of $\chi$}.
\end{mydef}

\begin{mydef}
  If $u_{\nu} = 0$ for all $r_{1}+1 \le \nu \le r_{2}$ and
  $v_{\nu}= 0$ for all $\nu$, in other words, if $\chi$ has infinity-type
  $(u_{1},\ldots, u_{r_{1}},0,\ldots, 0)$ then $\chi$ is called a \textbf{ray class
    character}.
\end{mydef}

Since $\chi$ is then trivial on the \emph{principal ray $P_{\mf m'}$}
corresponding to some modulus $\mf m' = \mf m \mf m_{\infty}$,
where $\mf m_{\infty}$ is determined by $\{u_{\nu} : 1\le \nu \le r_{1}\}$, $\chi$
descends to a character on the ray class group, $\bar \chi\colon \Cl_{\mf
  m'} \to \C^{\times}$. We will later restrict our attention to Hecke
characters of this type. 

\begin{eg}
  Let $\chi$ be any Dirichlet character. Then $\chi \circ \Nm_{F/\Q}$ is a Hecke
  character of $F$, called the \emph{base change of $\chi$ to $F$}.
\end{eg}

\begin{eg}\label{eg:Jacobi-sum}
  Fix $m \in \N_{>1}$. For a non-zero prime ideal $\mf p$ in $\Q(\zeta_{m})$ coprime to
  $m$ with $p \defeq \Nm \mf p$, and fixing $x \notin \mf p
  $, let $\chi_{\mf p }(x)$ be the root of unity in $\Q(\zeta_{m})$
  satisfying
  \begin{equation}
    \label{eq:24}
    \chi_{\mf p}(x) \equiv x^{\frac{p-1}{m}} \mod{\mf p}.
  \end{equation}
If $a = (a_{j}) \in (\Z/m\Z)^{r}$ is any non-zero vector, then we can
define
\begin{equation}
  \label{eq:25}
  J_{a}(\mf p) \defeq (-1)^{r+1} \sum_{\substack{x_{1},\ldots, x_{r} \mod \mf
      p \\ \sum x_{j} \equiv -1 \mod \mf p}}\chi_{\mf p}(x_{k})^{a_{j}}.
\end{equation}
Weil showed in \cite{weil1952} that $J_{a}$ is in fact a Hecke
character modulo $(m^{2})$ on $\Q(\zeta_{m})$.
\end{eg}

As with Dirichlet characters, we extend $\chi$ to ideals $\mf a$ not coprime to
$\mf m$ by setting $\chi (\mf a) = 0$.

The observation leading to the perhaps strange-looking definition of a
Hecke character was the following: if $F = \Q(\sqrt{D})$ with class
number $h$ and $\mf a \leq \O_{F}$ is an integral ideal, then we can fix $\alpha
\in F^{\times}$ such that $(\alpha) = \mf a^{h}$. Then
\begin{equation}
  \label{eq:27}
  \chi (\mf a) \defeq \begin{cases}
   \exp(\frac{\pi i}{\log |\epsilon|} \left|\frac{\alpha}{\alpha'}\right|) \qq{} &\text{if }D>
   0,\\
   \qty(\frac{\alpha}{|\alpha|})^{2} \qq{}&\text{if } D< 0,
 \end{cases}
\end{equation}
where $\epsilon$ is a fundamental unit of $F$ and $\alpha'$ is the conjugate of
$\alpha$, defines a character on the set of ideals of $\O_{F}$. Hecke
proved that these admit $L$-functions which behave like Dirichlet $L$-functions.
 \begin{mydef}
   Fix a Hecke character $\chi$. The \textbf{Hecke $L$-function} attached
   to $\chi$ is given by the Dirichlet series
   \begin{equation}
     \label{eq:18}
     L(\chi,s) \defeq \sum_{\mf a \leq \O_{F}} \chi(\mf a)\Nm(\mf a)^{-s}.
   \end{equation}
 \end{mydef}

 \begin{eg}
  If $\chi$ is the trivial character of modulus $\mf m = (1)$, then $L(\chi,s) = \zeta_{F}(s)$.
\end{eg}
\begin{eg}
  Returning to \cref{eg:Jacobi-sum}, Weil proved that the
  $L$-functions $L(J_{a},s)$ agree with the Hasse-Weil $L$-functions
  of a family of hyperelliptic curves, establishing a special case of the
  Hasse-Weil conjecture. This strategy of proving meromorphy of
  Hasse-Weil $L$-functions was later adopted by Deuring to
  CM elliptic curves, and by Shimura and Taniyama to CM abelian
  varieties. Some explicit examples are worked out in
  \cite[Ch.18]{ireland1990}.

  A key aspect of the characters (pun intended) appearing in this
  story is that they are \emph{algebraic}, which means they naturally
  occur as one-dimensional complex representations of the absolute
  Galois group of a number field. The 
  so-called \emph{Weil group}, an ``enlargement'', seems to encompass
  also the non-algebraic ones.
\end{eg}

{\color{red} There seems to be an analogy between Hecke characters of
  ``complicated'' infinity-type (i.e. non-algebraic, or perhaps even
  non-ray class characters) and Maa\ss\ forms; perhaps ask James about
this?} {\color{green} Related:
\href{https://mathoverflow.net/questions/221813/non-algebraic-hecke-characters}{SO
post} -- the answer says non-alg Hecke chars are analogous to non-alg
Maa\ss\ forms, i.e. those of eigenvalue $> 1/4$.}

 \begin{prop}
   Let $\chi$ be a Hecke character, and $L(\chi,s)$ the associated Hecke $L$-function.
\begin{enumerate}
\item $L(\chi,s)$ converges uniformly and
  absolutely on the half-plane $\Re(s)> 1 + \epsilon$ for any $\epsilon > 0$; 
\item $L(\chi,s)$ has an Euler product
  \begin{equation}
    \label{eq:21}
    L(\chi,s) = \prod_{\mf p }\frac{1}{1 - \chi(\mf p)\Nm(\mf p)^{-s}},
  \end{equation}
  where $\mf p$ runs over prime ideals of $\O_{F}$.
\item Let
  \begin{equation}
    \label{eq:Hecke-fn-eq}
    \Lambda(\chi,s ) \defeq \mqty(\frac{2^{r_{1}}|\Delta_{F}|\Nm(\mf m)}{(2\pi)^{g}})
    \prod_{\nu =1}^{r_{1}+r_{2}}\Gamma\qty(\frac{n_{\nu}s +|u_{\nu}| + i v_{\nu}}{2})L(\chi,s),
  \end{equation}
  where $n_{\nu} = 1$ if $1 \le \nu \le r_{1}$ and $n_{\nu}= 2$ otherwise. Then
  $\Lambda$ extends to a meromorphic on $\C$, and satisfies
  \begin{equation}
    \label{eq:23}
    \Lambda(\chi,1-s) = T(\chi)\Lambda(\bar \chi,s),
  \end{equation}
  for some $T(\chi) \in \C$ depending only on $\chi$ which satisfies $|T(\chi)| =
  1$.
\item If $\chi$ is trivial, then $\Lambda$ is holomorphic except for poles at
  $s=0$ and $s=1$; otherwise $\Lambda$ is entire. 
\end{enumerate}
\end{prop}
\begin{proof}
  See \cite[\S 3.3]{miyake1989}.
\end{proof}
 (iii) was originally proved using a
  method similar to Riemann's, namely by realising $L(\chi,s)$ as the
  Mellin transform of some theta function attached to $\chi$,
  generalising the classical theta function
  $\theta(x) \defeq \sum_{n \in \Z}e^{2\pi i n^{2}}$. This was recast more
  conceptually in the language of adèles by Tate in his famous PhD
  thesis, see \cite[Ch.XV]{cassels1967}.

  In the language of adèles, a Hecke character is nothing but a
  character on the idèle class group.

  
\subsection{$L$-functions of totally real fields}
It is natural to try to extend the constructions of the previous
section from $\Q$ to a general number field $F$. Suppose $\chi$ is a
ray class character over a field which admits both a real and a complex
embedding. By \cref{eq:Hecke-fn-eq}, the functional equation then has
the form {\color{red} fix this}
\begin{equation}
  \label{eq:2}
\Lambda(chi,s)  = \epsilon \Lambda(\bar \chi,1-s), \qq{}  \Lambda(\chi,s) =
C(\chi)^{s}\Gamma\qty(\frac{s}{2})^{a}\Gamma\qty(\frac{s+1}{2})^{b}L(\chi, s),
\end{equation}
where $|\epsilon| =1$, $C(\chi)$ is some constant depending on $\chi$, and
$a,b \in \Q$. Here $a = 0$ (resp. $b$) is non-zero iff $F$ has a real
(resp. imaginary) embedding into $\C$. We can rearrange this to obtain
an expression for $L(chi,1-s)$ in terms of $L(\chi,s)$. For $k\gg 0$, the
values $L(\chi,k)$ are finite and easily seen to be non-zero. However,
since $\Gamma$ has poles at the negative integers, the occurrence of both
real and complex embeddings implies that the values $L(\chi,1-k)$ all
vanish unless $F$ has either only real or only non-real embeddings
into $\C$. Accordingly we can only hope to $p$-adically interpolate
$L$-functions attached to totally real or totally imaginary number
fields.  cf {\color{red}Cassels Frollick? Probably Tate's thesis}

Fix $F$ a totally real field, $\psi \colon \Cl_{\mf m}(F) \to \C^{\times}$ a
ray class character and $p$ a prime. We define the \emph{Hecke $L$-function
  associated to $\psi$} by
\begin{equation}
  \label{eq:1}
  L(\psi,s) \defeq \sum_{\substack{\mf a \leq \O_{F}\\ (\mf a, \mf m) =1}} \psi(\mf a)\Nm(\mf a)^{-s},
\end{equation}
where by $\psi(\mf a)$ we mean $\psi$ evaluated at the class of $\mf a$ in
$\Cl_{\mf m}$. Hecke showed that $L(\psi,s)$ has an Euler product. 
It was possibly known to Hecke, and at any rate to Siegel and Klingen,
that the special values $L(\psi, 1-k)$ are rational when $k$ is an even
integer.
{\color{red} TODO: Write out the argument maybe?}
This was later proved by Shintani using analytical methods.
Removing the Euler factors above $p$, the resulting special values
\begin{equation}
  \label{eq:3}
L_{p}(\psi,1-k) \defeq  \prod_{\mf p \mid (p)}(1- \psi(\mf p) \Nm(\mf p )^{k-1}) L(\psi,1-k)
\end{equation}
interpolate $p$-adically by the work of Deligne-Ribet
\cite{deligne1980} to a $p$-adic analytic function $L_{p}(\psi,s)\colon
\Z_{p}\to \C_{p}$, or alternatively by the work of Barsky and
Cassou-Nogues \cite{cassou-nogues1979}, which is based on Shintani's method.

\begin{eg}
{\color{red}Perhaps it's better to write this out in detail?  }

Taking $F = \Q$, $\mf m = (1)$ and $\psi$ trivial recovers the
construction of the Kubota-Leopoldt $p$-adic $L$-function $\zeta_{p}$,
which interpolates $(1-p^{k-1})\zeta(1-k)$. 
\end{eg}
\subsection{$p$-adic interpolation}
\label{sec:p-adic-interpolation}
The main way of representing $L_{p}(\psi,s)$  is as a power series over
$\Q_{p}$. From a computational point of view, it is natural to
approximate this by polynomials. To avoid heavy arithmetic over
$\Q_{p}$, it is convenient to use Newton's divided differences
method; Lagrange interpolation would require divison by elements of
$\Z_{p}$, which could lead to precision loss.



\subsection{The algorithm}
\label{sec:algorithm}
We now describe the algorithm for computing a polynomial approximation
of $L_{p}(\psi,s)$. 

\begin{algorithm}[H]
\DontPrintSemicolon
  
\KwInput{
  \begin{itemize}[itemsep=-4pt]
  \item $F$ a totally real number field of degree $d$,
  \item $p$ an odd prime,
  \item $\psi \colon \Cl_{\mf m}\to \C^{\times}$ a totally odd or even Hecke character of $F$ of
    modulus $\mf m$,
  \item $k_{0}\in [2,p]$ with same parity as $\psi$,
  \item $m \in \N$.
\end{itemize}}
  \KwOutput{$P(s) \in \Z_{p}[q]/(p^{\delta_{m}},q^{m})$ approximating
    $L_{p}(\psi,s)$}\;
  % \KwData{Testing set $x$}
  % \tcp*{this is a comment}
  % \tcc{Now this is an if...else conditional loop}
  $\Psi,M \gets \mathtt{Diricihlet}(\psi)$ \tcp*{Described in subroutine X} \;
  $\delta_{m} \gets m\frac{p-1}{p-2}$ \;
  $S \gets $ Sturm bound for $M_{dk_{j}}(\Gamma_{0}(\Psi))$ \;\;
  \For{$j = 0,\ldots, \delta_{m}$}{
    $k_{j} \gets k_{0} +j(p-1)$ \; $M_{dk_{j}} \gets$ basis for
    $M_{dk_{j}}(\Gamma_{1}(M))$ mod $q^{S}$\; \tcc{Described in subroutine
      Y}
  $\Delta_{j}(q) \gets 2^{d}\sum_{n =1}^{S-1}\sum_{\nu \in \mf d_{+}^{-1}}\sum_{\mf a \mid
    (\nu)\mf d}\psi(\mf a) \Nm(\mf a)^{k_{j}-1}q^{n}$ \;
  \tcc{Described in subroutine Z} \;
  $L(\psi, 1-k_{j}) \gets $\texttt{find\_const\_term}$(\Delta_{j},M_{dk_{j}})$ \;
$L_{p}(\psi,1-k_{j}) \gets L(\psi, 1-k_{j}) \cdot \prod_{\mf p \mid (p)}(1-\psi(\mf p)\Nm(\mf
p)^{k_{j}-1})$}\;
$P(s) \gets $ \texttt{Newton\_poln}$\{\big(1-k_{j},L_{p}(\psi,1-k_{j})\big) : j = 0,\ldots \delta_{m}\}$


   % \If{Condition 1}
   %  {
   %      Do something    \tcp*{this is another comment}
   %      \If{sub-Condition}
   %      {Do a lot}
   %  }
   %  \ElseIf{Condition 2}
   %  {
   %  	Do Otherwise \;
   %      \tcc{Now this is a for loop}
   %      \For{sequence}    
   %      { 
   %      	loop instructions
   %      }
   %  }
   %  \Else
   %  {
   %  	Do the rest
   %  }
    
   %  \tcc{Now this is a While loop}
   % \While{Condition}
   % {
   % 		Do something\;
   % }

\caption{Compute $L_{p}(\psi,s)$}
\end{algorithm}

\begin{eg}
  Let $F = \Q(\sqrt 5)$, $p = 3$ and $\mf m = (4)$. Then
  $\Cl_{\mf m}^{+} \cong \Z/2\times \Z/2$, with generators
  $\mf a = (41,\frac{13+\sqrt{5}}{2})$ and $\mf b =
  (11,\frac{15+\sqrt{5}}{2})$. We fix the totally odd character $\psi$ on
  $\Cl_{\mf m}^{+}$ defined by $\psi(\mf a) =1$ and $\psi(\mf b) =
  -1$. Running the algorithm above with $m=8$ and $k_{0} = 3$ then gives
  \begin{align*}
    P(s) =\ldots + \left(3^{11} \cdot 2 + O(3^{12})\right)
    &s^{16} + \left(3^{9} \cdot 22 + O(3^{12})\right)
    s^{15} + \left(3^{10} \cdot 2 +O(3^{12})\right)
    s^{14}  + \left(3^{8} \cdot 10 + O(3^{12})\right)
    s^{13}\\
    + \left(3^{9} \cdot 5 + O(3^{12})\right)
    &s^{12} + \left(3^{7} \cdot 97 + O(3^{12})\right)
    s^{11}    + \left(3^{7} \cdot 157 + O(3^{12})\right)
    s^{10} + \left(3^{5} \cdot 914 + O(3^{12})\right)
    s^{9} \\
    + \left(3^{7} \cdot 98 + O(3^{12})\right)
    &s^{8} + \left(3^{5} \cdot 976 + O(3^{12})\right)
    s^{7} + \left(3^{6} \cdot 598 + O(3^{12})\right)
    s^{6} + \left(3^{4} \cdot 20 + O(3^{12})\right)
    s^{5} \\
    + \left(3^{4} \cdot 19067 + O(3^{13})\right)
    &s^{4} + \left(3^{2} \cdot 369067 + O(3^{14})\right)
    s^{3} + \left(3^{3} \cdot 185023 + O(3^{15})\right)
    s^{2}\\ + \left(3 \cdot 1380869 + O(3^{16})\right) 
    &s + 3^{17} \cdot 2 + O(3^{18})
  \end{align*}
  We note 1) that the $3$-adic valuation of the $n$-th coefficient of
  $P(s)$ is roughly $n$ so the general term tends to zero in $3$-adic
  absolute value, and 2) that the high power of $3$ dividing the
  constant term is likely explained by $L_{3}(\psi,0) =0$. As in
  \cite[Example 3.4]{lauder2021}, this simply reflects the fact that
  $3$ is inert in $\Q(\sqrt 5)$, so $\psi(3) =1$ as $\psi$ is trivial on
  principal ideals. The Euler factor at $p$ is
  $(1-1*\Nm(p)^{1-1}) =0$, and so $L_{p}(\psi,0)=0$. The branch of
  $k_{0}=1$ similarly gives
  \[ P(s) = \ldots +  \left(3^{2} \cdot 191920 + O(3^{14})\right) s^{3} + \left(3^{3} \cdot 185023 + O(3^{15})\right) s^{2} + \left(3 \cdot 10946807 + O(3^{16})\right) s
  \]
  with a clear zero at $s=0$.\footnote{The omission of $O(3^{18})$ in
    the branch at $1$ comes from the fact that the special value at
    $1$ is $0$, whereas for the branch at $3$ we didn't actually
    compute this value.}


  
  Similarly, for $p=5$ which is ramified in $F$, we see that
  \[ P(s) = \ldots + \left(5^{3} \cdot 142289 + O(5^{11})\right) s^{3} + \left(5^{2} \cdot 1002202 + O(5^{11})\right) s^{2} + \left(5 \cdot 19298281 + O(5^{12})\right) s + O(5^{13})
  \]
\end{eg}



\subsection{Lambda-invariants}
\label{sec:lambda-invariants}

\subsubsection{A quick summary of EJV}
\label{sec:quick-summary-ejv}
Here we should talk about \cite{ellenberg2011}. Also mentioned by
\cite{roblot2013}, better check this out

\subsubsection{Mention results of Santato?}
\label{sec:ment-results-sant}



Let $K / \Q_{p}$ be a finite extension, and $\O_{K}$ its ring of
integers, with uniformiser $\varpi$. The structure of power series
over $K$ is well-understood by the following theorem:

\begin{thm}[Weierstra\ss\ preparation theorem]
  Let $f \in \O_{K}\ps{T}$ be a power series. 
  Then $f$ factors as 
  \begin{equation}
    \label{eq:14}
    f(T)= \varpi^{\mu} P(T)u(T),
  \end{equation}
  where $\mu \in \Z_{\geq 0}$, $P(T) \in \O_{K}[T]$ and $u(T) \in \O_{K}\ps{T}^{\times}$.
\end{thm}
The number $\mu$ is called the \emph{(Iwasawa) $\mu$-invariant} of $f$,
and $\lambda \defeq \deg P(T)$ its \emph{$\lambda$-invariant}.

The $\lambda$-invariants of $p$-adic $L$-functions are particularly
interesting because of their connection to Iwasawa theory. For the
$\mu$-invariant, we have the following theorem:
\begin{thm}[Ferrero-Washington]
Suppose $F$ is a totally real abelian number field, $p$ is prime and $\chi$ a ring
class character of $F$. Then the $\mu$-invariant of the associated
$p$-adic $L$-function $L_{p}(\chi,s)$ is $0$. 
\end{thm}
For a reference, see \cite[\S10]{lang1990}. 

{\color{red} Give some examples of non-abelian extensions with
  non-zero $\mu$?}

The theorem was conjectured by Iwasawa, and he also gave a
counterexample to show that we cannot expect the result to hold for
arbitrary fields:
\begin{eg}
  
\end{eg}


\section{Implementation}
\label{sec:implementation}
We now describe how to interpolate $L_{p}(\psi,s)$ (somewhat)
efficiently, at least in the case of $F = \Q(\sqrt{D})$ a real
quadratic field. The ticket
\href{https://trac.sagemath.org/ticket/15829}{``Ray class groups and
  Hecke characters ``} brings experimental support for Hecke
characters to \texttt{sage}, allowing us in particular to evaluate
Hecke characters at ideals explicitly. However, this remains
inefficient in practice, since in the computation of the higher
Fourier coefficients of the diagonal restriction we need to perform
this evaluation a very large number of times. For real quadratic
fields, we can circumvent this with a method described in \cref{sec:quadr-forms-meth}.
\subsection{An algorithm for quadratic fields}

\subsection{The quadratic forms method}
\label{sec:quadr-forms-meth}
If we take $\psi$ to be a \emph{ring class character}, i.e. a character
of $\Cl_{(f)}^{+}(F)$, then we can use the classical reduction theory
of indefinite quadratic forms to find the coefficients of the diagonal
restrictions much more efficiently. 

One can show that if $\psi$ is totally even (resp. odd), then $\psi$ is a
linear combination of functions $\mathbbm{1}_{\mf a} +
\mathbbm{1}_{\mf b}$ (resp. $\mathbbm{1}_{\mf a} -
\mathbbm{1}_{\mf b}$) where
\begin{equation}
  \label{eq:4}
  \mathbbm{1}_{\mf a} \colon \Cl_{(f)}^{+} \to \{\pm 1\}, \qq{}  \mathbbm{1}_{\mf a}(\mc C) = \begin{cases}
    1 \qq{if} [\mf a] = \mc C, \\
    0 \qq{otherwise.}
  \end{cases}
\end{equation}
If we have a decomposition $\psi = \sum_{\mf a,\mf b}\mathbbm{1}_{\mf
  a}\pm\mathbbm{1}_{\mf b}$ for some fixed sign $\pm$, then we can
compute $\Delta_{j}$ by splitting the sum into partial sums corresponding
to fixed classes in the narrow class group. 

[[[ Can we do this automatically?]]]
\begin{eg}
  Let $F = \Q(\sqrt{5})$
\end{eg}

\subsubsection{Computing sets of nearly reduced forms}
\label{sec:comp-sets-nearly}

\emph{Compute $\mbb I(n,\mc C)$ using $\mbb I(d,\mc C)$ for some $d | n$}:
Input:
- $n$ a positive integer.
% - \texttt{forms_old}, a list of pairs \texttt{[Q_old,gamma_old]}, where \texttt{Q_old} is a quadratic form of discriminant $d^2D$, and \texttt{gamma_old} a "Hecke matrix" of determinant d

\begin{enumerate}
\item For $m = n/d$, compute matrices of the form
  \begin{equation}
    \label{eq:5}
    \mqty(m & j \\ 0 & m/d'), \qq{} 0 < j < d',
  \end{equation}
  for all $d' || m$.
\item For each


\end{enumerate}


\emph{Finding nearly reduced forms:} This step has a natural
interpretation in terms of the so-called Conway polytope: given our
indefinite quadratic form $Q_{0} =Q|_{\gamma}$, we first perform Gaussian
reduction to obtain a reduced form, which necessarily lies on the
\emph{river}. Now we apply translation and add each translate to our
list of nearly reduced forms iteratively, until we reach another
reduced form. When this happens, we reflect, and then continue translating.
We continue the process of translating and reflecting until we return
to our first reduced form, at which point the algorithm terminates,
and we return the list nearly reduced forms.


If $F = \Q(\sqrt{D})$, $Q = \langle a,b,c \rangle$ is an indefinite quadratic form with
positive root $\tau = \frac{-b+\sqrt{D}}{2a}$ for some
integers $a,b,c \in \Z$, then $Q$ determines an embedding of $\O_{F}$ into
$M_{2}(\Z)$ by $\alpha(\sqrt{D}) = \mqty(-b & -2c \\ 2a & a)$; indeed,
\[ \mqty(-b & -2c \\ 2a & a)^{2} = \mqty(b^{2}-4ac & 2bc - 2bc \\ -2ab
  + 2ab & -4ac + b^{2}) = D\mqty(1 & 0 \\ 0 & 1).
\]
We view this as an embedding of the quadratic order $\O_{F}$ into the
maximal order $M_{2}(\Z)$ of the split quaternion algebra
$M_{2}(\Q)$. For any $x \in F$, the action
\begin{equation}
  \label{eq:11}
  x \cdot M = \mqty(1 & 0 \\ 0 & 0) M\cdot \alpha(x) + \mqty(0 & 0 \\ 0 & 1) M\cdot
  \alpha(\bar x),
\end{equation}
makes $M_{2}(\Q)$ a $2$-dimensional vector space over $F$. 
\begin{lemma}
  There is a natural decomposition  $M_{2}(F) = M_{+}\oplus M_{-}$, where
  \begin{equation*}
    % \label{eq:12}
    M_{+} = \{M \in M_{2}(F) \colon x \cdot M = xM \} \qq{and}     M_{-} =
    \{M \in M_{2}(F) \colon x \cdot M = \bar x M \}.
  \end{equation*}
\end{lemma}
(Q: is this just the eigenspace decomposition?)
Now let $B\colon M_{2}(\Q) \to F$ be composition $M_{2}(\Q) \hookrightarrow M_{2}(F)
\xrightarrow{\pr_{+}} M_{+} \xrightarrow{\det} F$. 
\begin{prop}
  The map $B\colon M_{2}(\Q) \to F$ satisfies
  \begin{enumerate}
  \item $B(x\cdot M) = x^{2}B(M)$,
  \item $\tr_{F/\Q}B(M) = \det M$,
  \end{enumerate}
  and is uniquely characterised by these properties.
\end{prop}
Recall that for $\alpha \in F$, we write $\alpha \gg 0$ if $\alpha$ is \emph{totally
  positive}, that is, if $\sigma(\alpha)>0$ for all embeddings $\sigma \colon F \hookrightarrow \R$.
\begin{prop}
  Let $F = \Q(\sqrt{D})$ be a real quadratic field and $\mc C \in
  \Cl^{+}$ a fixed class in its narrow class group. Let
  \begin{equation}
    \label{eq:9}
  \mbb I(n,\mc C) \defeq \qty{(\mf a, \nu) :\hspace{5pt} \parbox[c]{2.7cm}{\raggedright $ \nu \in
        \mf d^{-1},\ \nu \gg 0$,\\ $\tr(\nu)= n$ \\ $\mf a \mid (\nu)\mf
        d,\ [\mf a] = \mc C $ }}
  \end{equation}
and 
  \begin{equation}
    \label{eq:8}
    M(n,\mc C) \defeq \qty{\gamma \in M_{2}(\Z)/\alpha(\O_{F}^{\times})
      :\hspace{3pt} \parbox[l]{1.9cm}{\raggedright $ \det \gamma = n$ \\
        $\det_{F}(\gamma) \gg 0$ }}.
  \end{equation}
  Then there is a bijection $\mbb I(n,\mc C) \leftrightarrow M(n,\mc C)$. 
\end{prop}

\subsection{Optimisations of the naïve algorithm}
\label{sec:optim-naive-algor}
\begin{itemize}
\item Parallelising the $\Delta$-step
\item Randomised bases in sage
\item Storing values of $\psi(\mf p)$ in a dictionary (introduces
  additional complexity of shared memory between processes)
\item Compare with magma algorithm?
\item Speed up computations by working mod $p^{N}$; should actually
  matter once we go big
\item Discussion of why this isn't good enough for statistics:
  \begin{itemize}
  \item computing mfs with large level on $\Gamma_{1}$ is still very
    inefficient in \texttt{sage} because Sturm bound gets large;
    forces computation of MANY coeffs
  \item 
    
  \end{itemize}

\end{itemize}


\section{Future investigations}
\label{sec:future-research}

% \subsection{Half-integer weight Hilbert modular forms}
% \label{sec:half-integer-weight}
% The prototypical example of a modular form of half-integer weight is
% the classical theta function,
% \begin{equation}
%   \label{eq:6}
%   \theta(z) = \sum_{n \in \Z}e^{2\pi izn^{2} } = 1 + 2\sum_{n=1}^{\infty} e^{2\pi i zn^{2}}.
% \end{equation}
% We have the functional equation $\theta(\gamma z) = j(\gamma,z) \theta(z)$ for $z \in \mf h$
% and $\gamma = \mqty(a & b \\ c & d) \in \Gamma_{0}(4)$, where
% \begin{equation}
%   \label{eq:7}
%   j(\gamma,z) \defeq \qty(\frac{c}{d}) \epsilon^{-1}_{d}\sqrt{cz+d}, \qq{} \epsilon_{d}
%   \defeq \begin{cases}
%     1 \qq{if} d \equiv 1 \mod 4 \\
%     i \qq{if} d \equiv 3 \mod 4
%   \end{cases}
% \end{equation}
% is known as the \emph{theta multiplier}. Based on this, for any odd
% integer $k$ a holomorphic function which transforms like a $k$-th power of
% $\theta$ is called \emph{a modular form of weight $k/2$}. More generally,
% if we fix a Dirichlet character $\chi\colon (\Z/N\Z)^{\times}\to \C^{\times}$, a
% modular form of weight $k/2$ and nebentypus $\chi$ is a function $f$
% holomorphic on $\mf h$ and ``at the cusps'' which satisfies
% \begin{equation}
%   \label{eq:15}
%   f(\gamma z) = \chi(d) j(\gamma,z)f(z) \qq{for all} \gamma = \mqty(a & b \\ c & d) \in \Gamma_{0}(N).
% \end{equation}
% Good references for this are \cite{serre1977} and \cite{koblitz2012}.
% Denote by $M_{k/2}(\Gamma_{0}(N))$ the space of modular forms with trivial
% nebentypus. The \emph{Kohnen $+$-space $M_{k/2}^{+}(\Gamma_{0}(N))$} is the subspace of
% $M_{k/2}(\Gamma_{0}(N))$ consisting of forms with Fourier expansion
% $\sum_{n=0}^{\infty}a(n)q^{n}$ where $a(n) = 0$ if $(-1)^{k-1/2} \not \equiv 0,1
% \pmod{4}$. Cohen proved that there is a unique Eisenstein series
% $\mc H_{k}(z) = \sum_{n=0}^{\infty}a_{k}(n)q^{n}$ in $M_{k/2}^{+}(\Gamma_{0}(N))$, and
% this has the property that if $D = (-1)^{k-1/2}n$ then $a_{k}(n) =
% L(\chi_{D},3/2-k)$. Here $\chi_{D} = \qty(\frac{\bullet}{D})$ is the quadratic
% character attached to the number field $\Q(\sqrt{D})$.


\bibliographystyle{alpha}
\bibliography{/home/havard/Documents/bibliography/references}

\end{document}
%%% Local Variables:
%%% mode: latex
%%% TeX-master: t
%%% End:
